\documentclass[a4paper,11pt]{article}

\usepackage[T1]{fontenc}
\usepackage[utf8]{inputenc}
\usepackage{graphicx}
\usepackage{xcolor}

\renewcommand\familydefault{\sfdefault}
\usepackage{tgheros}
\usepackage[defaultmono]{droidmono}

\usepackage{amsmath,amssymb,amsthm,textcomp}
\usepackage{enumerate}
\usepackage{multicol}
\usepackage{tikz}

\usepackage{geometry}
\geometry{left=25mm,right=25mm,%
bindingoffset=0mm, top=20mm,bottom=20mm}


\linespread{1.3}

\newcommand{\linia}{\rule{\linewidth}{0.5pt}}

% custom theorems if needed
\newtheoremstyle{mytheor}
    {1ex}{1ex}{\normalfont}{0pt}{\scshape}{.}{1ex}
    {{\thmname{#1 }}{\thmnumber{#2}}{\thmnote{ (#3)}}}

\theoremstyle{mytheor}
\newtheorem{defi}{Definition}

% my own titles
\makeatletter
\renewcommand{\maketitle}{
\begin{center}
\vspace{2ex}
{\huge \textsc{\@title}}
\vspace{1ex}
\\
\linia\\
\@author \hfill \@date
\vspace{4ex}
\end{center}
}
\makeatother
%%%

% custom footers and headers
\usepackage{fancyhdr}
\pagestyle{fancy}
\lhead{}
\chead{}
\rhead{}
\lfoot{Base de datos}
\cfoot{}
\rfoot{Page \thepage}
\renewcommand{\headrulewidth}{0pt}
\renewcommand{\footrulewidth}{0pt}
%

% code listing settings
\usepackage{listings}
\lstset{
    language=Python,
    basicstyle=\ttfamily\small,
    aboveskip={1.0\baselineskip},
    belowskip={1.0\baselineskip},
    columns=fixed,
    extendedchars=true,
    breaklines=true,
    tabsize=4,
    prebreak=\raisebox{0ex}[0ex][0ex]{\ensuremath{\hookleftarrow}},
    frame=lines,
    showtabs=false,
    showspaces=false,
    showstringspaces=false,
    keywordstyle=\color[rgb]{0.627,0.126,0.941},
    commentstyle=\color[rgb]{0.133,0.545,0.133},
    stringstyle=\color[rgb]{01,0,0},
    numbers=left,
    numberstyle=\small,
    stepnumber=1,
    numbersep=10pt,
    captionpos=t,
    escapeinside={\%*}{*)}
}

\begin{document}

\title{Normalización de Bases de datos}

\author{Eduardo Ríos, UNAM}

\date{19/03/2020}

\maketitle

Teniendo en cuenta el tema principal el cual ha sido las bases de datos, uno de sus componentes más importantes es la normalización la cual es de suma importancia para que su estructura este bien definida. Teniendo en cuenta esta información podemos decir que la normalización de las bases de datos es que: "Es el proceso de organizar los datos de una base de datos, valga la redundancia”.\\

Debemos tener en cuenta la creación de tablas y las reglas que se usan para definir las relaciones, estas reglas son diseñadas para proteger los datos, y para que la base de datos sea flexible con el fin de eliminar redundancias y dependencias incoherentes.\\

Las bases de datos relacionales se normalizan para:
\begin{itemize}
    \item Evitar la redundancia de los datos.
    \item Disminuir problemas de actualización de los datos en las tablas.
    \item Proteger la integridad de los datos.
    \item Facilitar el acceso e interpretación de los datos.
    \item Reducir el tiempo y complejidad de revisión de las bases de datos.
    \item Optimizar el espacio de almacenamiento.
    \item Prevenir borrados indeseados de datos.\\
\end{itemize}
Para que las tablas de la base de datos estén normalizadas deben cumplir las siguientes reglas:
\begin{itemize}
\item Cada tabla debe tener su nombre único.
\item No puede haber dos filas iguales.
\item No se permiten los duplicados.
\item Todos los datos en una columna deben ser del mismo tipo.\\
\end{itemize}

Teniendo en cuenta esto, debemos de conocer las tres primeras formas normales de normalizar, es preciso mencionar que existen más, pero estas antes mencionadas son las mas importantes. A continuación, las mencionare:


\section*{Primera forma normal}

Se expresa generalmente en forma de dos indicaciones separadas.
\begin{itemize}
\item Todos los atributos, valores almacenados en las columnas, deben ser indivisibles.
\item No deben existir grupos de valores repetidos.
\end{itemize}

\section*{Segunda forma normal}

Añade la necesidad de que no existan dependencias funcionales parciales. Esto significa que todos los valores de las columnas de una fila deben depender de la clave primaria de dicha fila, entendiendo por clave primaria los valores de todas las columnas que la formen, en caso de ser más de una.\\

Las tablas que están ajustadas a la primera forma normal, y además disponen de una clave primaria formada por una única columna con un valor indivisible, cumplen ya con la segunda forma normal. Ésta afecta exclusivamente a las tablas en las que la clave primaria está formada por los valores de dos o más columnas, debiendo asegurarse, en este caso, que todas las demás columnas son accesibles a través de la clave completa y nunca mediante una parte de esa clave.


\section*{Tercera forma normal}

Indica que no deben existir dependencias transitivas entre las columnas de una tabla, lo cual significa que las columnas que no forman parte de la clave primaria deben depender sólo de la clave, nunca de otra columna no clave.

\end{document}
